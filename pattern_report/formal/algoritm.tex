\begin{spacing}{1}

\indent При запуске программы вызывается меню из 8 функций.\\
\indent 0: Программа останавливает работу.\\
\indent 1: Генерируется дерево из случайных чисел, кол-во которых вводится в консоли. 
Генерация происходит следующим образом: в цикле из n итераций создается случайное значение, затем 
спускаемся до листов, при этом сравнивая значения (если меньше, то спускаемся влево, если больше, 
то спускаемся вправо). После того, как мы дошли до листа дерева, 
создаем новый элемент со значением, которое мы сгенерировали ранее.\\
\indent 2: Рисует дерево;\\
\indent 3: Добавляет элемент в дерево; Действия анологичны действиям из генерации дерева.\\
\indent 4: Удаляет из дерева введенный элемент. Для этого производим по дереву поиск нужного элемента,
затем нужно рассмотреть три возможные ситуации. Если у узла нет дочерних узлов, 
то у его родителя нужно просто заменить указатель на 0. Если у узла есть только один дочерний узел, 
то нужно создать новую связь между родителем удаляемого узла и его дочерним узлом. Если у узла два дочерних узла, 
то нужно найти следующий за ним элемент, его правого потомка подвесить на место найденного элемента, 
а удаляемый узел заменить найденным узлом.\\
\indent 5: Выводит кол-во вершин в дереве. Для этого рекурсивно проходимся по всему дереву и считаем кол-во вершин;\\
\indent 6: Удаляет дерево. Для этого дерево приравнивается к нулю.\\
\indent 7: Выводит кол-во вершин, имеющих два поддерева; Для этого рекурсивно проходимся по дереву, 
параллельно проводя проверку вершины. Если слева и справа есть дочерний узел, и при этом каждый из этих узлов 
имеет хотя бы один лист или узел, то увеличиваем кол-во вершин на единицу.\\

\end{spacing}