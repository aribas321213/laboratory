\documentclass[12pt]{report}
\usepackage{styles/log-style}
\usepackage{setspace}  

\title{lab_report}
\author{Рамиль Титеев}
\date{September 2022}

\begin{document}
    \begin{titlepage}
        \begin{center}
            \large{Московский Aвиационный Институт}\\
            \large{(Национальный Исследовательский Университет)}\\
            \vspace{0.4in}
            \textbf{\LARGE{Факультет информационных технологий и прикладной математики}}\\
            \vspace{0.4in}
            \large{Кафедра вычислительной математики и программирования}\\
            \vspace{0.4in}
            \textbf{\LARGE{Лабораторная работa 1 по курсу ОOП:}}\\
            \textbf{\LARGE{основы программирования на языке С\#}}\\
        \end{center}
        \vspace{0.6in}
        \small{1.АГРЕГАЦИЯ ПО ССЫЛКЕ}\\
        \vfill
        \begin{flushleft}
                \large{ 
                    Работу выполнил:\\
                    М8О-205Б-21 \hspace{0.1in} 
                    Титеев Р.М. \hspace{0.3in}  
                    \tline{(\textit{подпись})}{\hspace{1in}} 
                    \hspace{0.3in} 
                    \tline{(\textit{вариант})}{\hspace{0.7in}}\\ 
                    Руководитель: \tline{(\textit{подпись})}{\hspace{1in}}/Кузнецова С.В. \\
                    Дата: \underline{\hspace{0.4in}} октября 2022\\
                }
        \end{flushleft}        
    \end{titlepage}

    \textbf{\large{Агрегация по ссылке}}\\
    \begin{center}
        \includegraphics[scale=0.5]{formal/screen_2.png}\\
    \end{center}
    \textbf{Текст программы}\\
    \begin{lstlisting}[language={[Sharp]C}]
% put your code here
\end{lstlisting}
    \vspace{0.4in}
    \textbf{Результат работы}\\
    \begin{center}
        \includegraphics[scale=0.5]{formal/screen_1.png}\\
    \end{center}
    \textbf{Пример использования}\\
    Данный метод можно использовать в качестве описания структуры работников в какой-либо компании. 
    Так например есть главный класс и это будет главный человек в компании. Сначала мы инициализируем всех сотрудников 
    данной компании(при этом инициализация происходит по иерархии снизу вверх). Затем мы привязываем сотрудников к своим начальникам.
    Кроме этого, мы можем использовать уже проинициализированного работника в других целях, 
    и при этом его св-ва будут сохранятся в старой компании.\\
    \textbf{Вывод}\\
    Объекты классов A, B, C, D , E, J, K существуют независимо друг от друга. 
    При этом, например связывание объекта а с объектами b, c происходит с помощью конструктора; 
    b, c – параметры для конструктора A.
    Аналогичным образом происходит связывание $b$ c $d$, $c$ c $e$, $j$ и $j$  c $k$. 
    Но при этом объекты могут быть уничтожены по отдельности, что нарушает целостность структуры.

\end{document}
