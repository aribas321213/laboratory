\documentclass[12pt]{report}
\usepackage{styles/log-style}
\usepackage{setspace}  

\title{lab_report}
\author{Рамиль Титеев}
\date{September 2022}

\begin{document}
    \begin{titlepage}
        \begin{center}
            \large{Московский Aвиационный Институт}\\
            \large{(Национальный Исследовательский Университет)}\\
            \vspace{0.4in}
            \textbf{\LARGE{Факультет информационных технологий и прикладной математики}}\\
            \vspace{0.4in}
            \large{Кафедра вычислительной математики и программирования}\\
            \vspace{0.4in}
            \textbf{\LARGE{Лабораторная работa 2 по курсу ОOП:}}\\
            \textbf{\LARGE{основы программирования на языке С\#}}\\
        \end{center}
        \vspace{0.6in}
        \small{2.АГРЕГАЦИЯ ПО ЗНАЧЕНИЮ И ВЛОЖЕНИЕМ }\\
        \vfill
        \begin{flushleft}
                \large{ 
                    Работу выполнил:\\
                    М8О-205Б-21 \hspace{0.1in} 
                    Титеев Р.М. \hspace{0.3in}  
                    \tline{(\textit{подпись})}{\hspace{1in}} 
                    \hspace{0.3in} 
                    \tline{(\textit{вариант})}{\hspace{0.7in}}\\ 
                    Руководитель: \tline{(\textit{подпись})}{\hspace{1in}}/Кузнецова С.В. \\
                    Дата: \underline{\hspace{0.4in}} октября 2022\\
                }
        \end{flushleft}        
    \end{titlepage}

    \textbf{\large{Агрегация по значению}}\\
    \begin{center}
        \includegraphics[scale=0.5]{formal/screen_1.png}\\  
    \end{center}
    \begin{center}
        \includegraphics[scale=0.5]{formal/screen_2.png}\\  
    \end{center}
    \textbf{Текст программы}\\
    \begin{lstlisting}[language={[Sharp]C}]
using System;
using System.Collections.Generic;
using System.Linq;
using System.Text;
using System.Threading.Tasks;

namespace lab_5
{
    public interface C
    {
        void mC();
        int fC();
    }

    public class E : C
    {
        private int e;
        public E()
        {
            Console.WriteLine("   create E()");
            this.e = 10;
        }
        public E(int e)
        {
            Console.WriteLine("   create E(int e) ");
            this.e = e;
        }
        public int fE() { return this.e; }

        public void mC() {
            this.e = 0;
        }
        public int fC() { return this.e * 10; }
    }

    public interface J : C
    {
        int fJ();
        void mJ();
    }

    public class K : E, J
    {
        private int a = 0;
        public K() {
            Console.WriteLine("   create K() ");
            this.a = this.fC() + this.fE(); 
        }

        public K(int e) : base(e)
        {
            Console.WriteLine("   create K(int e) ");
            this.a = this.fE()+1;
        }
        public int fJ() { return this.a; }
        public void mJ() {
            Console.WriteLine("K  k.mj()");
            this.a = this.fC() + this.fJ(); 
        }
    }

    class Program
    {
        static void Main(string[] args)
        {
            C c = null;
            c = new E();
            Console.WriteLine("E  c.fC() = {0}", c.fC());
            E e = new E(1);
            Console.WriteLine("E  e.fC() = {0}", e.fC());
            Console.WriteLine("E  e.fE() = {0}", e.fE());

            J j = null;
            j = new K();
            Console.WriteLine("K  j.fC() = {0}", j.fC());
            Console.WriteLine("K  j.fj() = {0}", j.fJ());
            j = new K(30);
            Console.WriteLine("K  j.fC() = {0}", j.fC());

            c = new K();
            Console.WriteLine("K  c.fC() = {0}", c.fC());

            c = new K(99);
            Console.WriteLine("K  c.fC() = {0}", c.fC());

            K k = new K();
            Console.WriteLine("K  k.fj() = {0}", k.fJ());
            k.mJ();
            Console.WriteLine("K  k.fj() = {0}", k.fJ());

            Console.ReadKey();
        }
    }
}
\end{lstlisting}
    \vspace{0.4in}
    \textbf{Результат работы}\\
    \begin{center}
        \includegraphics[scale=0.5]{formal/lab_2_1.png}\\
    \end{center}
    \textbf{Пример использования}\\
    Используя данный метод можно описать устройство компьютера. Так, например, при инициализации 
    самого компьютера создадутся все нужные компоненты(материнская плата, процессор, видеокарта и т.д).
    Однако в самой программе мы можем отдельно создать компонент этого компьютера.
    Например можно создать отдельно процессор или любую другую часть, и при этом не запускать инициализацию самого компьютера.\\
    \textbf{Вывод}\\
    При агрегации по значению, все объекты создаваемого класса существуют внутри него самого, 
    уничтожение их отдельно от объекта, частью которого они являются невозможно. 
    Их уничтожение происходит при уничтожение самого верхнего в иерархии объекта. 
    Например: $b$, $c$ – части $a$, они создаются в конструкторе $a$, а их уничтожение происходит при вызове деструктора $a$.\\

    \newpage
    \textbf{\large{Агрегация вложением}}\\
    \begin{center}
        \includegraphics[scale=0.7]{formal/screen_3.png}\\  
    \end{center}
    \textbf{Текст программы}\\
    \begin{lstlisting}[language={[Sharp]C}]
using System;
using System.Collections.Generic;
using System.Linq;
using System.Text;
using System.Threading.Tasks;

namespace lab_4_2
{
    public interface A{
        void mA();
        int fA();
    }

    public interface B{
        int fB();
        void mB();
    }

    public class C{
        public C() { this.v_1 = 33; }
        public int v_1 { set; get; }
        public int f()
        {
            Console.WriteLine("class C f() ");
            return 1;
        }
    }

    public class D : C, A, B{
        public D() { this.v_2 = 1; this.v_3 = 2; }
        protected int v_2 { set; get; }
        public int v_3 { set; get; }

        public void mA() { this.v_2 = this.v_3 + this.v_1; }
        public int fA() { return this.v_3 * 10; }

        public int fB() { return this.v_3 * (10-this.v_1); }
        public void mB() { this.v_2 = this.v_1*this.v_3+100; }
    }
    internal class Program{
        static void Main(string[] args)
        {
            A a = null;
            a = new D();
            a.mA();
            Console.WriteLine("    Combination");
            Console.WriteLine($"a.fa() = {a.fA()}");
            Console.WriteLine($"((A)b).f() =  {((D)a).f()}");

            C c = new C();
            c.f();
            c = new D();

            Console.ReadKey();

        }
    }
}

\end{lstlisting}
    \vspace{0.4in}
    \textbf{Результат работы}\\
    \begin{center}
        \includegraphics[scale=0.5]{formal/lab_2_2.png}\\
    \end{center}
    \textbf{Пример использования}\\
    Данный метод организации классов может быть использован для описание какого-нибудь игрового персонажа со сложной структурой тела.
    Так, например, главным в этой иерархии будет сам персонаж, а части его тела будут создаваться внутри этого класса. 
    Таким образом мы можем изменять характеристики персонажа через главный класс, и при этом части тела нельзя будет создать отдельно в программе, неописав самого персонажа.\\
    \textbf{Вывод}\\
    При агрегации вложением, объявление классов происходит внутри классов, стоящих выше их по иерархии. 
    Все объекты создаваемого класса существуют внутри него самого, уничтожение их отдельно от объекта, 
    частью которого они являются невозможно. Их уничтожение происходит при уничтожение самого верхнего в иерархии объекта. 
    Например: $b$, $c$ – части $a$, они создаются в конструкторе $a$, а их уничтожение происходит при вызове деструктора $a$.\\


\end{document}
