\documentclass[12pt]{report}
\usepackage{styles/log-style}
\usepackage{setspace}  

\title{lab_report}
\author{Рамиль Титеев}
\date{September 2022}

\begin{document}
    \begin{titlepage}
        \begin{center}
            \large{Московский Aвиационный Институт}\\
            \large{(Национальный Исследовательский Университет)}\\
            \vspace{0.4in}
            \textbf{\LARGE{Факультет информационных технологий и прикладной математики}}\\
            \vspace{0.4in}
            \large{Кафедра вычислительной математики и программирования}\\
            \vspace{0.4in}
            \textbf{\LARGE{Лабораторная работa 4 по курсу ОOП:}}\\
            \textbf{\LARGE{основы программирования на языке С\#}}\\
        \end{center}
        \vspace{0.6in}
        \small{4. НАСЛЕДОВАНИЕ: РАСШИРЕНИЕ, СПЕЦИФИКАЦИЯ, СПЕЦИАЛИЗАЦИЯ, КОНСТРУИРОВАНИЕ, КОМБИНИРОВАНИЕ }\\
        \vfill
        \begin{flushleft}
                \large{ 
                    Работу выполнил:\\
                    М8О-205Б-21 \hspace{0.1in} 
                    Титеев Р.М. \hspace{0.3in}  
                    \tline{(\textit{подпись})}{\hspace{1in}} 
                    \hspace{0.3in} 
                    \tline{(\textit{вариант})}{\hspace{0.7in}}\\ 
                    Руководитель: \tline{(\textit{подпись})}{\hspace{1in}}/Кузнецова С.В. \\
                    Дата: \underline{\hspace{0.4in}} октября 2022\\
                }
        \end{flushleft}        
    \end{titlepage}

    \textbf{\large{Расширение, Спецификация, Специализация, Конструирование}}\\
    \textbf{Текст программы}\\
    \begin{lstlisting}[language={[Sharp]C}]
using System;
using System.Collections.Generic;
using System.Linq;
using System.Text;
using System.Threading.Tasks;

namespace lab_5
{
    public interface C
    {
        void mC();
        int fC();
    }

    public class E : C
    {
        private int e;
        public E()
        {
            Console.WriteLine("   create E()");
            this.e = 10;
        }
        public E(int e)
        {
            Console.WriteLine("   create E(int e) ");
            this.e = e;
        }
        public int fE() { return this.e; }

        public void mC() {
            this.e = 0;
        }
        public int fC() { return this.e * 10; }
    }

    public interface J : C
    {
        int fJ();
        void mJ();
    }

    public class K : E, J
    {
        private int a = 0;
        public K() {
            Console.WriteLine("   create K() ");
            this.a = this.fC() + this.fE(); 
        }

        public K(int e) : base(e)
        {
            Console.WriteLine("   create K(int e) ");
            this.a = this.fE()+1;
        }
        public int fJ() { return this.a; }
        public void mJ() {
            Console.WriteLine("K  k.mj()");
            this.a = this.fC() + this.fJ(); 
        }
    }

    class Program
    {
        static void Main(string[] args)
        {
            C c = null;
            c = new E();
            Console.WriteLine("E  c.fC() = {0}", c.fC());
            E e = new E(1);
            Console.WriteLine("E  e.fC() = {0}", e.fC());
            Console.WriteLine("E  e.fE() = {0}", e.fE());

            J j = null;
            j = new K();
            Console.WriteLine("K  j.fC() = {0}", j.fC());
            Console.WriteLine("K  j.fj() = {0}", j.fJ());
            j = new K(30);
            Console.WriteLine("K  j.fC() = {0}", j.fC());

            c = new K();
            Console.WriteLine("K  c.fC() = {0}", c.fC());

            c = new K(99);
            Console.WriteLine("K  c.fC() = {0}", c.fC());

            K k = new K();
            Console.WriteLine("K  k.fj() = {0}", k.fJ());
            k.mJ();
            Console.WriteLine("K  k.fj() = {0}", k.fJ());

            Console.ReadKey();
        }
    }
}
\end{lstlisting}
    \vspace{0.4in}
    \newpage
    \textbf{Результат работы}\\
    \begin{center}
        \includegraphics[scale=0.5]{formal/lab_4_1.png}\\
    \end{center}

    \newpage
    \textbf{\large{Комбинирование}}\\
    \textbf{Текст программы}\\
    \begin{lstlisting}[language={[Sharp]C}]
using System;
using System.Collections.Generic;
using System.Linq;
using System.Text;
using System.Threading.Tasks;

namespace lab_4_2
{
    public interface A{
        void mA();
        int fA();
    }

    public interface B{
        int fB();
        void mB();
    }

    public class C{
        public C() { this.v_1 = 33; }
        public int v_1 { set; get; }
        public int f()
        {
            Console.WriteLine("class C f() ");
            return 1;
        }
    }

    public class D : C, A, B{
        public D() { this.v_2 = 1; this.v_3 = 2; }
        protected int v_2 { set; get; }
        public int v_3 { set; get; }

        public void mA() { this.v_2 = this.v_3 + this.v_1; }
        public int fA() { return this.v_3 * 10; }

        public int fB() { return this.v_3 * (10-this.v_1); }
        public void mB() { this.v_2 = this.v_1*this.v_3+100; }
    }
    internal class Program{
        static void Main(string[] args)
        {
            A a = null;
            a = new D();
            a.mA();
            Console.WriteLine("    Combination");
            Console.WriteLine($"a.fa() = {a.fA()}");
            Console.WriteLine($"((A)b).f() =  {((D)a).f()}");

            C c = new C();
            c.f();
            c = new D();

            Console.ReadKey();

        }
    }
}

\end{lstlisting}
    \vspace{0.4in}
    \textbf{Результат работы}\\
    \begin{center}
        \includegraphics[scale=0.5]{formal/lab_4_2.png}\\
    \end{center}
    \textbf{Вывод}\\
    При специализации дочерний класс является более конкретным, частным или специализированным случаем родительского класса.\\
    Расширение дает возможность добавить новые функциональные возможности к родительскому классу, но не меняет наследуемое поведение.\\
    При спецификации родительский класс описывает поведение, которое реализуется в дочернем классе, но оставлено нереализованным в родительском.\\
    Во время конструирование дочерний класс использует методы, предопределяемые родительским классом.\\
    Комбинирование позволяет объединить черты нескольких классов(интерфейсов) в одном (в данном случае трех).\\

\end{document}
