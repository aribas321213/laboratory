\begin{spacing}{1}

\indent На вход программе подается файл содержащий необходимые данные. В них входят размер прямоугольной программы, 
количество ненулевых элементов, и сами данные матрицы представленные в трех векторах. 
Парралельно вводу матрицы происходит поиск модуля максимального значения\\
\indent После того как произойдет считываение всех данных, программы начинает работу. Сначала выводится матрица в виде трех векторов, 
какими она была введена, затем она преобразуется в прямоугольную матрицу и тоже выводится в терминал.\\
\indent Затем начинается преобразование. Проходимся по строкам прямоугольной матрицы, и если встретилось число, 
по модулю равное максимальному, то делим всю строку на это значение.\\
\indent Затем выводим преобразованную прямоугольную матрицу, конвертируем её в сокращеный вид и выводим его тоже.


\end{spacing}