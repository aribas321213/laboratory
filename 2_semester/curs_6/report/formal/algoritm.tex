\begin{spacing}{1}

\indent При запуске $dump.cpp$ нужно ввести два названия файлов: входной файл, в котором находятся изначальные данные, и выходной, в который будет произведена запись. Затем создается двоичный файл и в него производится запись. \\
\indent После этого запускаем вторую программу $cols.cpp.$ Для неё нужно ввести сначала ключ, который определит, 
что требуется от программы, а затем два названия файлов: двоичный файл, который был создан предыдущей программой, 
и выходной $txt$-файл, в который будет произведена запись. Далее проверяем ключ: $-f$ значит, что нужно просто вывести 
все данные в виде таблицы и записать их в файл, $-t$ значит, что нужно вывести данные, удовлетворяющие заданному условию.\\
\indent Для условия $-f$ все очевидно: просто проходимся по файлу и выводим все его данные.\\
\indent Для $-t$ почти аналогично: считанные данные проверяем на выполнения условия, в зависимости от этого записываем его в файл и выводим в таблицу
в терминале или пропускаем. После всех проверок в терминале выводится готовая таблица, а в файле записываются
все данные, которые удовлетворяют условию. 


\end{spacing}