\documentclass[10pt]{report}

\usepackage{styles/log-style}
\usepackage{setspace}

\begin{document}
    % header
    \begin{spacing}{1.7}
        \hspace{1.2in}
        \textbf {Отчет по Лабораторной работе} № 24\\     % указать номер лабораторной
        \indent \hspace{1.2in} по курсу: \underline{1 фундаментальная информатика } \\
        \indent \hspace{1.2in} студент группы : \underline{М8О-105Б-21 Титеев Рамиль Маратович }, № по списку: \underline{ 23 } \\
        \indent \hspace{1.2in} Адреса www, e-mail, jabber, skype: \underline{ derol.gym@gmail.com } \\
        \indent \hspace{1.2in} Работа выполнена: "2 мая 2022г" \\   % указать дату
        \indent \hspace{1.2in} Преподаватель: \underline{ каф. 806 В.К.Титов} \\
        \indent \hspace{1.2in} Входной контроль знаний с оценкой: \tlinee{2in} \\
        \indent \hspace{1.2in} Отчет сдан "\tlinee{0.2in}"\tlinee{1in}20\tlinee{0.2in}г., итоговая оценка \tlinee{1in}\\
        \indent \hspace{1.2in} Подпись преподавателя \tlinee{1.2in}\\
    \end{spacing}
    
    \begin{spacing}{1}
        \indent \textbf{1. Тема}: \underline{Алгоритмы и структуры данных.}\\               % Ввести тему
        \indent \textbf{2. Цель работы}: \underline{Составить программу выполнения заданных преобразований арифметических}\\
        \indent \underline{выражений с применением деревьев.} \\   % Ввести цель работы      % Ввести цель работы   
        \indent \textbf{3. Задание}(\textit{вариант 5}): \underline{Упростить выражения, выполнить возведение числа в степень}\\
        \indent \underline{с положительным показателем.}\\% Ввести задание
        \indent \textbf{4. Оборудование}(\textit{лабораторное}):  \\ 
        \indent ЭВМ \tlinee{0.5in}, процессор \tlinee{0.7in}, имя узла сети \tlinee{0.3in} с ОП \tlinee{0.2in} ГБ\\
        \indent НМД \tlinee{0.5in} ГБ. Терминал \tlinee{0.7in} адрес \tlinee{0.7in}. Принтер \tlinee{0.7in}\\
        \indent Другие устройства \tlinee{2.5in} \\
        \indent \textit{Оборудование ПЭВМ студента, если использовалось}:  \\ 
        \indent Процессор \underline{ Ryzen 4600 @ 6x 3.0 GHz }, ОП \underline{ 16384 } МБ, НМД \tlinee{0.5in}ГБ. Монитор \underline{ Встроенный }\\
        \indent Другие устройства \tlinee{2.5in} \\

        \indent \textbf{5. Программное обеспечение}(\textit{лабораторное}):  \\ 
        \indent Операционная система семейства UNIX, наименование \underline{\hspace{0.5in}} версия \underline{\hspace{0.5in}}\\
        \indent Интерпретатор  команд: \underline{\hspace{2in}} версия \tlinee{0.5in}\\
        \indent Система программирования: \tlinee{2in} версия \tlinee{0.5in}\\
        \indent Редактор текстов: \tlinee{2in} версия \tlinee{0.5in}\\
        \indent Утилиты операционной системы: \tlinee{4in}\\
        \indent Прикладные системы и программы: \tlinee{3.7in}\\
        \indent Местонахождение и имена файлов и программ данных: \tlinee{2.2in}\\

        \indent \textit{Программное обеспечение ЭВМ студента, если использовалось}:  \\ 
        \indent Операционная система семейства UNIX, наименование \underline{ Ubuntu } версия \underline{ 20.04 }\\
        \indent Интерпретатор  команд: \underline{ bash } версия \tlinee{0.5in}\\
        \indent Система программирования: \underline{ C } версия \tlinee{0.5in}\\
        \indent Редактор текстов: \underline{ Emacs } версия \tlinee{0.5in}\\
        \indent Утилиты операционной системы: \tlinee{4in}\\
        \indent Прикладные системы и программы: \tlinee{3.7in}\\
        \indent Местонахождение и имена файлов и программ данных: \underline{ /usr/bin , a также /bin }\\

    \end{spacing}
    \newpage
    \begin{spacing}{1}
        \indent \textbf{6. Идея, метод, алгоритм} решения задачи (в формах: словесной, псевдокода, графической [блок-схема, диаграмма, рисунок, таблица] или формальное описание с пред- и постусловиями)\\
        \begin{spacing}{1}

\indent При запуске программы вызывается меню из 8 функций.\\
\indent 0: Программа останавливает работу.\\
\indent 1: Генерируется дерево из случайных чисел, кол-во которых вводится в консоли. 
Генерация происходит следующим образом: в цикле из n итераций создается случайное значение, затем 
спускаемся до листов, при этом сравнивая значения (если меньше, то спускаемся влево, если больше, 
то спускаемся вправо). После того, как мы дошли до листа дерева, 
создаем новый элемент со значением, которое мы сгенерировали ранее.\\
\indent 2: Рисует дерево;\\
\indent 3: Добавляет элемент в дерево; Действия анологичны действиям из генерации дерева.\\
\indent 4: Удаляет из дерева введенный элемент. Для этого производим по дереву поиск нужного элемента,
затем нужно рассмотреть три возможные ситуации. Если у узла нет дочерних узлов, 
то у его родителя нужно просто заменить указатель на 0. Если у узла есть только один дочерний узел, 
то нужно создать новую связь между родителем удаляемого узла и его дочерним узлом. Если у узла два дочерних узла, 
то нужно найти следующий за ним элемент, его правого потомка подвесить на место найденного элемента, 
а удаляемый узел заменить найденным узлом.\\
\indent 5: Выводит кол-во вершин в дереве. Для этого рекурсивно проходимся по всему дереву и считаем кол-во вершин;\\
\indent 6: Удаляет дерево. Для этого дерево приравнивается к нулю.\\
\indent 7: Выводит кол-во вершин, имеющих два поддерева; Для этого рекурсивно проходимся по дереву, 
параллельно проводя проверку вершины. Если слева и справа есть дочерний узел, и при этом каждый из этих узлов 
имеет хотя бы один лист или узел, то увеличиваем кол-во вершин на единицу.\\

\end{spacing}
    
        \indent \textbf{7. Сценарий выполнения работы} [план работы, первоначальный текст программы в черновике (можно на отдельном листе) и тесты, либо соображения по тестированию].\\
        
\begin{spacing}{1}

    \lstset{language=C++,
            basicstyle=\ttfamily,
            keywordstyle=\color{blue}\ttfamily,
            stringstyle=\color{red}\ttfamily,
            commentstyle=\color{green}\ttfamily,
            morecomment=[l][\color{magenta}]{\#}
    }
    \indent \large{\underline{lab\_24.cpp:}}
    \inputminted{c++}{../lab_24.cpp}
    
    
\end{spacing}
        \textit{Пункты 1-7 отчѐта составляются \textbf{строго до} начала лабораторной работы.}\\
        \begin{flushright}
            Допущен к выполнению работы. Подпись преподавателя \underline{\hspace{1.5in}}\\
        \end{flushright}
    
        \indent \textbf{8. Распечатка протокола} (подклеить листинг окончательного варианта программы с текстовыми
        примерами, подписанный преподавателем).\\
        \verbatiminput{formal/log.txt}
    
        \indent \textbf{9. Дневник отладки} должен содержать дату и время сеансов отладки, и основные ошибки (ошибки в
        сценарии и программе, не стандартные операции) и краткие комментарии к ним. В дневнике отладки
        приводятся сведения об использовании других ЭВМ, существенном участии преподавателя и других
        лиц в написании и отладке программы.\\
        \begin{center}
    \begin{tabular}{ | c | c | c | c | c | c | c | }
        \hline
        № & Лаб или дом. & Дата & Время & Событие & Действие по исправлению & Примечание \\ \hline
        \hspace{0cm} & \hspace{0cm} & \hspace{0cm}& \hspace{0cm}& \hspace{0cm}& \hspace{0cm}& \hspace{0cm} \\
        \hline
    \end{tabular}
\end{center}

        
        \indent \textbf{10.} Замечание автора по существу работы \underline{\hspace{3.5in}}\\
        \underline{\hspace{5.5in}}
        
        \indent \textbf{11.} Выводы \underline{\hspace{0.1in}Я научился составлять программы для преобразований арифметических выражений}\\
        \underline{с применением деревьев.\hspace{0.2in}}\\


        Недочеты, допущенные при выполнении задания, могут быть устранены следующим образом \underline{\hspace{0.5in}}\\
        \underline{\hspace{5.5in}}

        \begin{flushright}
            Подпись студента \underline{\hspace{1.5in}}\\
        \end{flushright}

    \end{spacing}
    
   
    
    
\end{document}