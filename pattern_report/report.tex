\documentclass[10pt]{report}

\usepackage{styles/log-style}
\usepackage{setspace}

\begin{document}
    % header
    \begin{spacing}{1.7}
        \hspace{1.2in}
        \textbf {Отчет по лабораторной работе} № \\     % указать номер лабораторной
        \indent \hspace{1.2in} по курсу: \underline{1 фундаментальная информатика } \\
        \indent \hspace{1.2in} студент группы : \underline{М8О-105Б-21 Титеев Рамиль Маратович }, № по списку: \underline{ 23 } \\
        \indent \hspace{1.2in} Адреса www, e-mail, jabber, skype: \underline{ derol.gym@gmail.com } \\
        \indent \hspace{1.2in} Работа выполнена: "ДД месяц 2022г" \\   % указать дату
        \indent \hspace{1.2in} Преподаватель: \underline{ каф. 806 В.К.Титов} \\
        \indent \hspace{1.2in} Входной контроль знаний с оценкой: \tlinee{2in} \\
        \indent \hspace{1.2in} Отчет сдан "\tlinee{0.2in}"\tlinee{1in}20\tlinee{0.2in}г., итоговая оценка \tlinee{1in}\\
        \indent \hspace{1.2in} Подпись преподавателя \tlinee{1.2in}\\
    \end{spacing}
    
    \begin{spacing}{1}
        \indent \textbf{1. Тема}: \\               % Ввести тему
        \indent \textbf{2. Цель работы}: \underline{}  \\       % Ввести цель работы
        \indent \textbf{3. Задание}: \underline{}  \\           % Ввести задание
        \indent \textbf{4. Оборудование}(\textit{лабораторное}):  \\ 
        \indent ЭВМ \tlinee{0.5in}, процессор \tlinee{0.7in}, имя узла сети \tlinee{0.3in} с ОП \tlinee{0.2in} ГБ\\
        \indent НМД \tlinee{0.5in} ГБ. Терминал \tlinee{0.7in} адрес \tlinee{0.7in}. Принтер \tlinee{0.7in}\\
        \indent Другие устройства \tlinee{2.5in} \\
        \indent \textit{Оборудование ПЭВМ студента, если использовалось}:  \\ 
        \indent Процессор \underline{ Ryzen 4600 @ 6x 3.0 GHz }, ОП \underline{ 16384 } МБ, НМД \tlinee{0.5in}ГБ. Монитор \underline{ Встроенный }\\
        \indent Другие устройства \tlinee{2.5in} \\

        \indent \textbf{5. Программное обеспечение}(\textit{лабораторное}):  \\ 
        \indent Операционная система семейства UNIX, наименование \underline{\hspace{0.5in}} версия \underline{\hspace{0.5in}}\\
        \indent Интерпретатор  команд: \underline{\hspace{2in}} версия \tlinee{0.5in}\\
        \indent Система программирования: \tlinee{2in} версия \tlinee{0.5in}\\
        \indent Редактор текстов: \tlinee{2in} версия \tlinee{0.5in}\\
        \indent Утилиты операционной системы: \tlinee{4in}\\
        \indent Прикладные системы и программы: \tlinee{3.7in}\\
        \indent Местонахождение и имена файлов и программ данных: \tlinee{2.2in}\\

        \indent \textit{Программное обеспечение ЭВМ студента, если использовалось}:  \\ 
        \indent Операционная система семейства UNIX, наименование \underline{ Ubuntu } версия \underline{ 20.04 }\\
        \indent Интерпретатор  команд: \underline{ bash } версия \tlinee{0.5in}\\
        \indent Система программирования: \underline{ C } версия \tlinee{0.5in}\\
        \indent Редактор текстов: \underline{ Emacs } версия \tlinee{0.5in}\\
        \indent Утилиты операционной системы: \tlinee{4in}\\
        \indent Прикладные системы и программы: \tlinee{3.7in}\\
        \indent Местонахождение и имена файлов и программ данных: \underline{ /usr/bin , a также /bin }\\

    \end{spacing}
    \newpage
    \indent \textbf{6. Идея, метод, алгоритм} решения задачи (в формах: словесной, псевдокода, графической [блок-схема, диаграмма, рисунок, таблица] или формальное описание с пред- и постусловиями)\\
    \begin{spacing}{1}

\indent Описание трансформации дерева:\\
\indent Находим вершину, которая содержит символ операции(+/-/*/...). 
Затем берем значения слева и справа от этой вершини и считаем его. 
После записываем его вместо вершины, в которой раньше находился символ 
операции. Однако если результат оказался больше 10 или меньше 0, 
то изменения не происходят.

\end{spacing}
   
    
    
\end{document}