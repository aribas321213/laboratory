\begin{spacing}{1}

\indent При запуске программы вызывается меню из 11 функций.\\
\indent 0: программа останавливает работу;\\
\indent 1: генерируется список из элементов, число которых вводится в консоли. 
Генерация происходит следующим образом: в цикле n раз создается новый элемент со случайным значением, 
указатель на следующий элемент последнего эелемента списка указывает на новый элемент, 
указатель нового эелемента на следующий указывает на первый элемент списка;\\
\indent 2: добавляется новый элемент в конец списка. Действия анологичны генерации списка, только 
значение принимается из консоли и выполняется это действие один раз;\\
\indent 3: добавляет элемент в начало списка; Происходит это следующим образом: 
создается новый элемент со значением, введенном из консоли. 
Указатель нового элемента на следующий укзывает на первый элемент прошлого списка, 
а указатель на на предыдущий элемент указывает на последний элемент прошлого списка.\\
\indent 4: Удаляет из списка введенный элемент. Для этого циклом проходимся по списку, 
пока не встретим нужное значение, а затем с помощьу указателей убираем его из списка, и потом 
удаляем данный элемент из памяти;\\
\indent 5: Выводит все элементы списка. Циклом проходим все элементы списка и выводим их значение;\\
\indent 6: Вставляет элемент после указанного. Происходит это следующим образом: 
циклом идем по списку, пока не встретим элемент, имеющий нужное значение, 
затем создаем новый элемент, указатель на следующий элемент у нового элемента указывает на элемент, 
который ранее шел после найденного, указатель на следующий элемент у найденного ранее элемента 
указываем на только что созданный элемент;\\
\indent 7: очищает список. Циклом прожодим по списку пока не дойдем до конца и удаляем из памяти каждый элемент;\\
\indent 8: выводит длину списка. Циклом прожодим по списку пока не дойдем до конца и считаем кол-во элементов;\\
\indent 9: выводит меню повторно;\\
\indent 10: вызывается особое действие над списком. Требуется ввести кол-во удаляемых элементов. 
Если число больше длины списка, то ничего не происходит и выводится ошибка. Если же оно меньше или равно, 
то в цикле который повторяется столько раз, сколько было введено, обращаемся к предпоследнему элементу через указатели,
указатель на следующий элемент выбранного элемента указываем на начало списка и удаляем из памяти последний элемент списка.\\


\end{spacing}